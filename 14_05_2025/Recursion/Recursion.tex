Parameterized recursion
functional recursion 
Great! Let's clearly understand the two important types of recursion often seen in programming:

---

## 🔁 1. **Parameterized Recursion**

### 📌 **Definition:**

In **parameterized recursion**, the function carries additional **parameters** (like counters, accumulators, or results) whose values **change across recursive calls**. These parameters **help carry state** through the recursive call stack.

---

### 📘 **Example: Sum of first N numbers**

```java
public static void sum(int i, int total) {
    if (i == 0) {
        System.out.println("Sum is: " + total);
        return;
    }
    sum(i - 1, total + i);  // Carrying sum in a parameter
}
```

### 🔍 How it works:

* Here, `i` controls the countdown.
* `total` accumulates the sum.
* This is **parameterized** because `total` holds the partial result.

---

## 🔁 2. **Functional Recursion (or Pure Functional Recursion)**

### 📌 **Definition:**

In **functional recursion**, we don’t carry state with parameters. Instead, we **solve the subproblem**, then **combine the result** using return values. Think of it like **a function returning an answer** instead of passing it forward.

---

### 📘 **Example: Sum of first N numbers**

```java
public static int sum(int n) {
    if (n == 0) return 0;
    return n + sum(n - 1);  // Combine result while returning
}
```

### 🔍 How it works:

* Each call returns a result.
* The function waits for the result from `sum(n - 1)` and adds `n` to it.
* This is **functional recursion** because we use return values to build the final answer.

---

## ✅ Summary Table:

| Feature                | Parameterized Recursion                     | Functional Recursion                       |
| ---------------------- | ------------------------------------------- | ------------------------------------------ |
| Uses extra parameters? | Yes (to carry results or counters)          | No                                         |
| Result mechanism       | Built while going deeper (forward pass)     | Built while returning (backward pass)      |
| Use-case               | Better for accumulations, printing          | Better for returning values                |
| Example task           | Printing sum, factorial (with side-effects) | Calculating sum, factorial, Fibonacci etc. |

---

### 🧠 Tip:

* Use **parameterized recursion** when:

  * You need to print or accumulate during the forward traversal.
  * You want better control over intermediate state.
  
* Use **functional recursion** when:

  * You just want a final return value.
  * You don’t need to track or print in-between states.

